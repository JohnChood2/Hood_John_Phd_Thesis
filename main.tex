\listfiles
\documentclass[12pt]{report}

\usepackage[intoc]{nomencl}
\textwidth=6in \oddsidemargin=0.5in \topmargin=-0.5in
\textheight=9in  % 9in must include page numbers
\textfloatsep = 0.4in \addtocontents{toc}{\vspace{0.4in} \hfill
Page\endgraf} \addtocontents{lof}{\vspace{0.2in} \hspace{0.13in} \
Figure\hfill Page\endgraf} \addtocontents{lot}{\vspace{0.2in}
\hspace{0.13in} \ Table\hfill Page\endgraf}

\usepackage{amsmath}
\usepackage{graphicx}
\usepackage{multirow}
\usepackage{caption}
\usepackage{setspace}
\usepackage{titlesec}
\usepackage{color}
\usepackage[left=1.5in,right=1in,top=1in,bottom=1in]{geometry}
\usepackage[table]{xcolor}
 \usepackage{amsfonts}
 \usepackage{amsmath}
 \usepackage{amsbsy,bm}
 \usepackage{amssymb}
\usepackage{graphicx}
 \usepackage{setspace}
 \usepackage{rotating}
 \usepackage{float}
 \usepackage{stmaryrd}
 \usepackage{multirow}
 \usepackage{color}
 \usepackage{soul}
 \usepackage{caption}
\usepackage{eepic}
\usepackage{colortbl}
\usepackage[numbers]{natbib}
\usepackage {multirow}
\usepackage{setspace}
\usepackage{indentfirst}
\usepackage{titlesec}
\usepackage{subfig}
\usepackage[mathscr]{euscript}
\usepackage[titletoc,title]{appendix}
\usepackage[titletoc]{appendix}
\usepackage[tocgraduated]{tocstyle}

\usepackage{textcomp}
\usepackage{array}
\usepackage{listings}
\usepackage{setspace}
\usepackage{mathptmx}
\usepackage{colortbl}
\usepackage{graphicx}
\usepackage{amssymb, amsmath}
\usepackage{subfig}
\usepackage{epsfig}
\usepackage{times}
\usepackage{float}
\usepackage{rotating}
\usepackage{makeidx}
\usepackage{url}
\usepackage{multirow}
\usepackage{booktabs}
\usepackage[subfigure, titles]{tocloft}
\usepackage{acronym}
\usepackage{datetime}
\usepackage{algorithm}
\usepackage{algorithmic}

\renewcommand{\nomname}{SPT, CMB-S4, VNA, TES, BOLOS, MKIDS}
\makenomenclature
\graphicspath{{images/}}
\DeclareGraphicsExtensions{.pdf,.jpeg,.png,.PNG, .eps, .tiff}

\urlstyle{same}

\usepackage{makecell}
\usepackage{titletoc}
\usepackage{appendix}
\usepackage[nottoc]{tocbibind}
\setcounter{secnumdepth}{7}
\setcounter{tocdepth}{7}


%new chapter/section and subsection commands
\newcommand{\hsuchapter}[1]{\chapter*{#1} \addcontentsline{toc}{chapter}{#1} } 
\newcommand{\hsusection}[1]{\section*{#1} \addcontentsline{toc}{section}{#1} } 
\newcommand{\hsusubsection}[1]{\subsection*{#1} \addcontentsline{toc}{subsection}{#1} } 

%%%%%%%Configure Table of Contents%%%%%%%%%%%%
\renewcommand{\contentsname}{TABLE OF CONTENTS}
\renewcommand{\cftchapfont}{\normalfont}
\renewcommand{\cftchappagefont}{\normalfont}
\renewcommand{\cftchapleader}{\cftdotfill{\cftdotsep}}

%%%%%%%Configure List of Figures%%%%%%%%%%%%
\renewcommand{\listfigurename}{LIST OF FIGURES}
\setlength{\cftbeforefigskip}{0.2in}

%%%%%%%Configure List of Tables%%%%%%%%%%%%
\renewcommand{\listtablename}{LIST OF TABLES}
\setlength{\cftbeforetabskip}{0.2in}

%%%%%%%Configure Bibliography%%%%%%%%%%%%
\renewcommand{\bibname}{ \texorpdfstring{{BIBLIOGRAPHY\vspace{10mm}}}{BIBLIOGRAPHY}   }

%%%%%%%Configure Chapter Headings%%%%%%%%%%%%
\makeatletter
\def\@makechapterhead#1{%
%  \vspace*{50\p@}%
  {\parindent \z@ \centering
    \normalfont
    \ifnum \c@secnumdepth >\m@ne
      \if@mainmatter
        \@chapapp\space \thechapter
        \par\nobreak
        \vskip 20\p@
      \fi
    \fi
    \interlinepenalty\@M
    #1\par\nobreak
    \vskip 40\p@
  }}
\def\@schapter#1{\if@twocolumn
                   \@topnewpage[\@makeschapterhead{#1}]%
                 \else
                   \@makeschapterhead{#1}%
                   \@afterheading
                 \fi}
\def\@makeschapterhead#1{%
 % \vspace*{-10\p@}%
  {\parindent \z@ \centering
    \normalfont
    \interlinepenalty\@M
    #1\par\nobreak
    \vskip 10\p@
  }}


%%%%%%%Configure Section Headings%%%%%%%%%%%%
\renewcommand\section{\@startsection {section}{1}{\z@}%
                                   {-3.5ex \@plus -1ex \@minus -.2ex}%
                                   {2.3ex \@plus.2ex}%
                                   {\centering\normalfont}}
                                   
%%%%%%%Configure Sub-Section Headings%%%%%%%%%%%%
\renewcommand\subsection{\@startsection {subsection}{2}{\z@}%
                                   {-3.5ex \@plus -1ex \@minus -.2ex}%
                                   {2.3ex \@plus.2ex}%
                                   {\noindent \normalfont  }}

%%%%%%%Sub-Sub-Section's Not  Supported%%%%%%%%%%%%

%%%%%%%Configure Table of Contents Heading%%%%%%%%%%%%
\renewcommand{\@cftmaketoctitle}{
  \chapter*{\contentsname}
  \addcontentsline{toc}{chapter}{TABLE OF CONTENTS}} 

%%%%%%%Configure List of Figures Heading%%%%%%%%%%%%
\renewcommand{\@cftmakeloftitle}{
  \chapter*{\listfigurename}
  Figure \hfill Page
  \addcontentsline{toc}{chapter}{LIST OF FIGURES} } 
  
%%%%%%%Configure List of Tables Heading%%%%%%%%%%%%
\renewcommand{\@cftmakelottitle}{
  \chapter*{\listtablename}
   Table \hfill Page
   \addcontentsline{toc}{chapter}{LIST OF TABLES} }  

\makeatother

\setcounter{section}{-1}     

% Define new commands here
\newcommand{\etal}{\emph{et al.}}
\newcommand{\leftsup}[2]{{\vphantom{#2}}^{#1}{#2}}
\newcommand{\leftsub}[2]{{\vphantom{#2}}_{#1}{#2}}
\newcommand{\leftsupsub}[3]{{\vphantom{#3}}^{#1}_{#2}{#3}}

\DeclareMathOperator*{\assembly}{\textbf{\Large A} }

\definecolor{lightblue}{rgb}{.90,.95,1} 
\newcommand{\hllb}[1]{
	\sethlcolor{lightblue}
	\hl{#1}
	\sethlcolor{yellow}
	}

\newcommand{\hlc}[2][yellow]{{\sethlcolor{#1}\hl{#2}} }

 % Define floats here
 \floatstyle{plain}
 \newfloat{Box}{h}{lob}
 \newcommand{\boxedtext}[3]{
 	\begin{Box} \caption{\small{#1}}
	\hspace{1.cm}
	\fbox{\begin{minipage}[c]{0.85\linewidth} 
	
	\small{#2}
       
       \end{minipage}}
       
       \label{#3}
       \end{Box}
  }

 \begin{document}

%\pagestyle{myheadings} \markright{\today}

\pagenumbering{alph}

\begin{titlepage}
\thispagestyle{empty}\enlargethispage{\the\footskip}%
\begin{center}
	{\setstretch{1.66} {Design and Implementation of a Multi-Frequency Polarization Sensitive Antenna}\par }%
	\vskip.4in
	By
	\vskip .3in
	{John C. Hood II}
	\vskip .3in
	
	\begin{doublespace}
	Dissertation\\
		Submitted to the Faculty of the \\
		Graduate School of Vanderbilt University \\
		in partial fulfillment of the requirements \\
		for the degree of \\ [.1in]
	\end{doublespace}
	
	\MakeUppercase{DOCTOR OF PHILOSOPHY} \\[.1in]
	in \\[.1in]
	{Astrophysics} \\[.25in]
	Spring 2021 \\[.25in]
	Nashville, Tennessee
	\vskip .5in
\end{center}
%%%Uncomment for Signatures%%%

Approved: \hskip 2.9in Date:\\[1.2em]
\rule{3.5in}{.5pt} \hskip 0.1in \rule{2in}{.5pt} \\[.01in]
[Committee Co-Chair]\\[.14in]
\rule{3.5in}{.5pt} \hskip 0.1in \rule{2in}{.5pt}  \\[.01in]
[Committee Co-Chair]\\[.14in]
\rule{3.5in}{.5pt} \hskip 0.1in \rule{2in}{.5pt} \\[.01in]
[Committee Member]\\[.14in]
\rule{3.5in}{.5pt} \hskip 0.1in \rule{2in}{.5pt} \\[.01in]
[Committee Member]\\[.14in]
\rule{3.5in}{.5pt} \hskip 0.1in \rule{2in}{.5pt} \\[.01in]
[Committee Member]\\[.14in]
\rule{3.5in}{.5pt} \hskip 0.1in \rule{2in}{.5pt} \\[.01in]
[Committee Member]\\[.14in]



%\\[.14in]
%%%%%%%%%%%%%%
%%%%%%Uncomment  for Approved Names%%%%%%
%\begin{center}
%\begin{doublespace}
%Approved:\\
%Professor John D. Doe \\
%Professor Jane D. Doe
%\end{doublespace}
%\end{center}
\end{titlepage}
 
\doublespacing
\pagenumbering{roman} \setcounter{page}{2}

\chapter*{The dedication page is optional. If you don't use it, please delete it.}
\addcontentsline{toc}{chapter}{DEDICATION}
\vspace{7mm}

\chapter*{ACKNOWLEDGMENTS}
\addcontentsline{toc}{chapter}{ACKNOWLEDGMENTS}
\vspace{7mm}
This page is optional. If you don't use it, please delete it.
\tableofcontents

\listoftables
\newpage
\listoffigures
\newpage


\normalsize
\doublespacing
\pagenumbering{arabic}
\setcounter{page}{1}

%---------------------------------------------------------------------------------------------------------------%
%------------------------------------------------SECTION------------------------------------------------%
%---------------------------------------------------------------------------------------------------------------%
\chapter{\bf{Cosmic Microwave Background}}
\vspace{-7mm}
\begin{figure}[h!]
\centering
\includegraphics[scale=1]{Hood_utilities/OLD_BACKUP/Pictures/cosmology_diagrams/Inflation_cmb_powspec_v4.png}   %\includegraphics[width=0.70\linewidth]{Inflation_cmb_powspec_v4.png}
  \caption{Current measurements of the B-mode spectrum are shown for the BICEP2/Keck Array (light orange), POLARBEAR (orange), and SPTPol (dark orange). The lensing contribution to the B-mode spectrum can be partially removed by measuring the E and exploiting the non-Gaussian statistics of the lensing.}
  \label{fig:current B-Mode measurements}
\end{figure}


\section{Introduction}\label{Introduction}

\begin{figure}[h!]
\centering
\includegraphics[scale=.5]{../Desktop/B-Bump.png} 
 \caption{Descriptive plot of the B-Bump seen in the E-modes and B-modes of the CMB power spectrum}
  \label{fig:B-Bump}
\end{figure}

\section{Anisotropies}\label{Anisotropies}

\section{Observatories}\label{Observatories}

\section{Conclusion}\label{Conclusion}

\clearpage

%---------------------------------------------------------------------------------------------------------------%
%------------------------------------------------SECTION------------------------------------------------%
%---------------------------------------------------------------------------------------------------------------%
\chapter{\bf{South Pole Telescope(SPT-3G)}}\label{SPT-3G}
\vspace{-7mm}

\section{Science Objectives}\label{Science Objectives}

\section{Instruments}\label{Instrument}


\section{Conclusion}\label{Conclusion}





\chapter{\bf{CMB-S4}}\label{CMB-S4}
\vspace{-7mm}

\section{Science Objectives}\label{Science Objectives}

\section{Instruments}\label{Instruments}


\section{Conclusion}\label{Conclusion}

Xxxxxx



\chapter{\bf{Detector Testing and Fabrication}}\label{Detector Testing and Fabrication}

\section{Introduction}
The CMB-S4 target sensitivity is $\sigma(r)\sim 10^{-3}$ and will require pristine measurements of the B-mode power spectra. These measurements will target the degree-scale features found at low $l$'s rater than the larger reionization features found in the tens-of-degree scales. In order to reach the required sensitivity for r in thees observations we will need to make a significant leap forward in raw instruments sensitivity. This is where we began to make progress in this work to ensure that the detector technology will help reach this goal. Here we will describe the detector technology used to make these improvements along with the fabrication and testing process.

\section{Ortho mode Transducers(OMT's)}
The OMT's designed for this project are made of four paddle shaped probes suspended on a silicon-nitride membrane inside a wave-guide. the wave-guide is terminated with a  measured at one quarter of the tuned wavelength beneath the probes.  Our OMT's are tuned for 100GHz in order to test the probes ability to couple a signal from a waveguide onto  grounded co-planar waveguide (GCPW) traces made of layered gold with an impedance of $\approx$ 50$\Omega$. 

\section{Test Materials and Set-Up}

\section{Warm Testing}

\section{Conclusion}


\chapter{\bf{Sub-Millimeter AGN Catalogue}}\label{Sub-Millimeter AGN Catalogue}

\section{Introduction}

\section{Observations}

\section{Maps}

\section{Point Sources}

\section{Conclusion}







\bibliographystyle{unsrt}
\bibliography{references_list}


\end{document} 


